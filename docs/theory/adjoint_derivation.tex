\documentclass[11pt]{article}
\usepackage{amsmath, amssymb, bm}
\usepackage{physics}

\begin{document}

\section*{Adjoint Method for Maxwell's Equations}

We derive the adjoint equations corresponding to the FDTD
discretization of Maxwell's equations in a dispersive medium.

\section{Forward Problem}

Maxwell's equations in a non-magnetic, dispersive material:
\begin{align}
    \nabla \times \mathbf{E} &= -\mu_0 \frac{\partial \mathbf{H}}{\partial t}, \\
    \nabla \times \mathbf{H} &= \frac{\partial \mathbf{D}}{\partial t},
\end{align}
where the constitutive relation is
\begin{equation}
    \mathbf{D}(\mathbf{r}, t) = \varepsilon(\mathbf{r}, \omega) \mathbf{E}(\mathbf{r}, t).
\end{equation}

Let the objective functional be
\begin{equation}
    S = \int_0^T \sum_m \abs{\mathbf{E}(\mathbf{r}_m(t), t)}^2 \, dt.
\end{equation}

\section{Lagrangian Formulation}

Define the Lagrangian
\begin{align}
    \mathcal{L} 
    &= S 
    + \int_0^T \int_V 
        \bm{\lambda}_E \cdot 
        \left( 
            \nabla \times \mathbf{H}
            - \frac{\partial \mathbf{D}}{\partial t}
        \right) \, dV \, dt \\
    &\quad + 
    \int_0^T \int_V 
        \bm{\lambda}_H \cdot 
        \left(
            \nabla \times \mathbf{E}
            + \mu_0 \frac{\partial \mathbf{H}}{\partial t}
        \right) 
        dV \, dt.
\end{align}

\section{Adjoint Equations}

By taking variations of $\mathcal{L}$ w.r.t.\ $\mathbf{E}$ and $\mathbf{H}$,
we obtain the adjoint Maxwell equations:

\subsection*{Adjoint electric field}

\begin{equation}
    -\nabla \times \bm{\lambda}_H 
    - \frac{\partial}{\partial t}
    \left(
        \varepsilon \, \bm{\lambda}_E
    \right)
    = \frac{\partial S}{\partial \mathbf{E}}.
\end{equation}

\subsection*{Adjoint magnetic field}

\begin{equation}
    \nabla \times \bm{\lambda}_E
    - \mu_0 \frac{\partial \bm{\lambda}_H}{\partial t}
    = 0.
\end{equation}

Boundary conditions follow the same absorbing (PML) structure as the forward problem,
but applied backwards in time.

\section{Gradient with Respect to Design Parameters}

The gradient of the objective with respect to a design parameter $\rho(\mathbf{r})$
is
\begin{equation}
    \frac{\partial S}{\partial \rho(\mathbf{r})}
    =
    -\int_0^T 
    \bm{\lambda}_E(\mathbf{r}, t)
    \cdot
    \frac{\partial \varepsilon(\mathbf{r}, \rho)}{\partial \rho}
    \mathbf{E}(\mathbf{r}, t)
    \, dt.
\end{equation}

\section*{Summary}

The adjoint fields $(\bm{\lambda}_E, \bm{\lambda}_H)$ evolve backward in time
with a source term located at molecular positions.
This formulation enables efficient gradient computation.

\end{document}
