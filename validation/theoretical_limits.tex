\documentclass[11pt]{article}
\usepackage{amsmath, amssymb}
\begin{document}

\section*{Validation of Theoretical Limits}

This document records the theoretical validation tests used to confirm the
correctness of the Hamiltonian formulation and the coupled EM--molecule model.

\section{1. Static Molecule Limit}

When the molecule is fixed ($\mathbf{r}_m(t) = \mathbf{r}_0$) and dipole variation
is ignored, the interaction Hamiltonian reduces to:
\begin{equation}
    H_{\text{int}} = -\bm{\mu} \cdot \mathbf{E}(\mathbf{r}_0).
\end{equation}
The EM Hamiltonian must then reproduce known plasmonic analytical results for:
\begin{itemize}
    \item a flat interface
    \item a nanosphere
    \item a nanodisk
\end{itemize}

\section{2. Free-Field Limit}

If the nanostructure is removed ($\varepsilon(\mathbf{r}) = \varepsilon_0$),
the EM field must propagate as:
\begin{equation}
    \mathbf{E}(\mathbf{r}, t) = \mathbf{E}_0 e^{i(\mathbf{k}\cdot \mathbf{r} - \omega t)}.
\end{equation}
This is used to verify:
\begin{itemize}
    \item CFL time step correctness
    \item Yee grid dispersion relation
\end{itemize}

\section{3. Weak Coupling Limit}

For small dipole moment:
\begin{equation}
    \abs{\bm{\mu}} \ll \abs{\bm{\mu}_0},
\end{equation}
the interaction Hamiltonian matches perturbation theory:
\begin{equation}
    \Delta S \propto \abs{\bm{\mu}}^2.
\end{equation}

\section{4. Energy Conservation}

In the absence of PML and loss:
\begin{equation}
    \frac{d}{dt} (H_{\text{EM}} + H_{\text{MD}} + H_{\text{int}}) = 0.
\end{equation}

\section{5. Comparison With Published Analytical Solutions}

We compare:
\begin{itemize}
    \item nanodisk plasmon resonance position
    \item dipole–field interaction strength
    \item alanine dihedral distribution
\end{itemize}

\end{document}
